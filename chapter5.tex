\newpage
\chapter{Methodology/ Approach/ Techniques}

\section{Architecture of Systems}

The staggering measure of information on the Internet is a rich asset for any field of
examination or individual premium. To successfully reap that information, you'll need to
turn into gifted at web scratching\\.

The scraping is done as follows:\\
1. Use demands and Beautiful Soup for scratching and parsing information from the Web\\
2. Stroll through a web scratching pipeline beginning to end\\
3. Assemble a content that brings an item detail including its value, rating, surveys, URL and
the item portrayal from the Web and shows important data in the comfort. Likewise, we
have utilized the mechanization office given by selenium which is a free (open-source)
robotized testing system used to approve web applications across various programs and
stages After viably collecting that information for example after web scratching we are
adding a voice help to it, to make it more easy to understand. As we probably are aware
Python is a reasonable language for script scholars and developers. The question for the
collaborator can be controlled according to the clients need. Discourse acknowledgment is
the way toward changing over sound into text. This is usually utilized in voice aides like
Alexa, Siri, and so on Python gives an API called Speech Recognition to permit us to change
over sound into text for additional preparing.\\

\section{Algorithm Used}
1. Pseudo -Relevance Feedback:\\
Pseudo relevance feedback , also known as blind relevance feedback , provides a method for
automatic local analysis. It automates the manual part of relevance feedback, so that the
user gets improved retrieval performance without an extended interaction. The method is
to do normal retrieval to find an initial set of most relevant documents, to then assume that
the top ranked documents are relevant, and finally to do relevance feedback as before
under this assumption.. The idea behind relevance feedback is to take the results that are
initially returned from a given query, to gather user feedback, and to use information about
whether or not those results are relevant to perform a new query. We can usefully
distinguish between three types of feedback: explicit feedback, implicit feedback, and blind
or ”pseudo” feedback.\\
2. Page Rank Algorithm:\\
PageRank (PR) is an algorithm used by Google Search to rank websites in their search engine
results. PageRank is a way of measuring the importance of website pages. PageRank works
by counting the number and quality of links to a page to determine a rough estimate of how
important the website is. The underlying assumption is that more important websites are
likely to receive more links from other websites.\\

\section{ Approach}
1) Information Gathering ( Using Web Scraping)\\
2) Preprocessing of gathered data\\
3) Query and retrieving\\
4) User Interface\\
5) Applying the most efficient algorithm for personalization (in our case we have used page
rank and pseudo relevance feedback)\\

\vspace{10mm}
\hrule
