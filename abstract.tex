\newpage
\addcontentsline{toc}{chapter}{Abstract}

\chapter*{Abstract\markboth{Abstract}{}}
Web site personalization is defined as the process of customizing the web-content and structure of a Web site to the specific and individual needs of each user taking advantage of the user’s navigational behavior.
 The main goal of personalization is delivering the content and thus matching the functionality or interests that user needs, with no effort from the targeted users. Personalization may emphasize user's particular information, grant or restrict access to certain tools, or simplify transactions and processes by remembering information about a user.\\
On a traveling web-site, a user may see advertisements, promotions and specials for locations they have visited before or recently searched for. On an internet, personalization could remove access to a tool intended only for certain employees. In an application, personalization might study user past searches so as to enable quick access of information that might be of interest for future to the users. In none of these instances do users need to take any action to make these changes: the system makes the call based on the identity of the user.The purpose of this project is to provide personalized experience to the user while searching data. Users, when searched data will inform about its relevance as a feedback and according a customized search, will be provided to the users. In our project we are going to implement a personalization system in which user will browse the content according to their requirement then from the searched information they will provide feedback for the relevance of that where the data will be processed through various stages as editorial planning, content reusing, navigation and content hierarchy.
                     

\vspace{10mm}
\hrule