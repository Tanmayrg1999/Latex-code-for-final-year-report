\newpage
\chapter{References}

\addcontentsline{toc}{chapter}{References}
\renewcommand{\bibname}{References}
%\nocite{*}

1.	Postma,O.,Brokke,M.Personalisation in practice:The proven effects of personalisation.J database Mark Cust strategy Manag 9,137-142(2002)/10.1057/palgrave(Springer)\\
2.	Magdalini Eirinaki and M. Vazirgiannis. “Web Mining for web personalisation” ACM Transactions on Internet Technology(2003):1-27doi:10.1145/634377.643478\\
3.	“Personalisation of web search results based on user profiling”, in proceedings of First International Conference on Emerging Trends in Engineering and technology(2008),IEEE,doi:10.1109/ICETET.2008.28\\
4.	Explicit User Profiles in Web Search PersonalisationProceedings of the 2011 15th International Conference on Computer Supported Cooperative Work in Design,IEEE,978-1-4577-0387-4/11 \\
5.	B. Mobasher, “Data Mining for Web Personalization”,The Adaptive Web: Methods and Strategies of WebPersonalization, Lecture Notes in Computer Science, NewYork, 2006, Vol. Springer-Verlag, Berlin-Heidelberg.\\
6.	O.R. Zaiane, “Building a Recommender Agent for e-Learning Systems”, in Proc. of the 7th International Conference on Computers in Education, Auckland, New Zealand, December, 2002, 3 – 6, pp 55-59.\\
7.	O. Nasraoui, “World Wide Web Personalization”,Invited chapter in “Encyclopedia of Data Mining and Data Warehousing”, 2005, J. Wang, Ed, Idea Group.\\
8.	Kardan, A., Fani, M.R. and Mohammadian, N. (2011), Proposing an Architecture for Learner Modeling based on Web Usage Mining in e-Learning Environment, The 5th Data Mining Conference Dec.14 - 15, 2011 ; Tehran, Iran\\
9.	Amirhossein Roshanzamir ,University of Bradford | UB · School of Management, Doctoral Student, Web Personalization; Implications and Challenges,ISCET\\
10.	Velaquez, J. D. and Palade, V. (2007), Building a knowledge base for implementing a web-based computerized recommendation system, International Journal on Artificial Intelligence Tools,Oct2007, Vol. 16 Issue 5, pp. 798.\\
11.	Pramila M. Chawan, Veermata Jijabai Technological Institute | VJTI · Department of Computer Technology M.E. COMPUTER ENGINEERING, Web usage mining : A research area in Web mining, Symposium on computer engineering and technology (ISCET).\\



\bibliographystyle{ieeetran}
\bibliography{references} 