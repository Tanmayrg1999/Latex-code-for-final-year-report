\newpage
\chapter{User Documentation }
Website Personalization is the process of creating customized experiences for
visitors to a website. Rather than providing a single, broad experience, website
personalization allows companies to present visitors with unique experiences tailored
to their needs and desires.\\
The major benefits of Personalization include –\\
1. Online retailers can provide targeted offers to shoppers based on browsing
behaviour.\\
2. Travel sites can present visitors with promotions based on the current weather
or season.\\
3. News and other media outlets can surface specific videos to viewers based
on where they live.\\
The following things should be provided from the user side -\\
1) Location - The country, region, or city a user is located in.\\
2) Technology - A user’s device-type (desktop, mobile, tablet), operating system,
browser, and even screen solution.\\
3) Traffic Sources - The specific traffic source a user is visiting from, be it direct
or paid, via referral search or social.\\
4) 3 rd party data - Information about a user that has been collected from outside
sources and aggregated by a DMP.\\
5) Behaviour - Important user interactions such as clicks, add-to-carts, or
purchase events, as well as the number of page views, URLs visited, and so
on.\\
6) Explicit data - CRM data that has been collected about a user or has been
provided intentionally through surveys and registration forms.\\
7) Time- The select dates, days of the week, or time of day the experience is to
be served to a user.\\
8) Current Page - The type of page a user lands on, whether its a specific URL,
the homepage, a product detail page (PDP), or cart page.
Users can refer to our Github Repository for user documentation –
https://github.com/Tanmayrg1999/Major-Project.git
