\newpage
\chapter{Literature Review}

1.\\ \\ A paper was published by M. Rami Ghorab, Dong Zhou,
Alexander O’Connor, Vincent Wade Published online on 13
May 2012 at the Springer Science Business which focused on
Personalized Information Retrieval survey and its
classification. It had a mechanism to learn about their
users’ search interests by either explicitly supplying this
information implicitly gathering this information in an
unobtrusive manner from the users’ search history.\\ In order to
provide a personalized service,their system maintained
information and data about the users and the history of their
interactions among user and system. This paper had a three-step
approach including the information gathering which included
how and which information to be gathered from the user and
his usage behavior. The next step included representation of
the gathered information either through vector-based models
or Semantic network-based Models where user’s interests are
maintained in a network structure of terms and its related
terms.User models here are represented using a semantic
network structure. In this case the model is made up of nodes
and associated nodes that capture terms and their
semantically-related or co-occurring terms respectively.The
next step would be implementation of the query adaption by
the use of various algorithmic approaches like global analysis,
Explicit relevance feedback, interactive query expansion.\\

2.\\ \\ Another paper written by Magdalini Erinaki in the year 2003
focused on Web mining for Web personalization. This
paper includes personalization techniques such as Content
based filtering-recommendation based on individual past
ratings and preferences and the Rule based filtering- based on
some predefined set of rules. The overall process of
web usage-based Web personalization basically has five different modules,
which correspond to each step of the process. These are as
follows.\\
a. User profiling\\
b. Log analysis and Web usage mining\\
c. Content management\\
d. Web site publishing\\
e. Information acquisition and searching\\
It merely records the addresses of pages requested by its user
thus highlighting interesting hyperlinks without involving the
user .The main advantage that they achieved was that the
search is user specific and the process is quite easier but Both
rule based and content based techniques do not provide large
scope and hence limiting benefits.A study was done on another paper published in IEEE in the
year 2008 which enlightened Profiling. They worked with
Implicit data gathering and personalization focusing on
Memory based technique which involves saving past rating for
each user for different items and maintaining pairwise
similarity between users and also the Model based technique
was used where Models are built on past searches and
prediction models are generated based on behavioral data
collection and Real time recommendation score generation.
But the major problem that they found here was that Data
Sparsity and inefficient user group categorization can lead to
unreliable results.\\ 

3.\\ \\ A further study was done on a paper published in IEEE 2011
written by Mandeep Pannu which focused on Explicit user
profiles in web Personalization. This paper includes
personalization techniques such as Vector based modeling
which involves creating preferences vectors for the users and
comparing them with the same vector model created for web
data and then comparing them. Also, they used the Probability
modeling which worked for estimating the probability of
relevance to the ranking document for a query.\\ The main
advantage of this paper was that this Allows ranking
documents according to their possible relevance but Long
documents are poorly represented because they have poor
similarity values. This implemented system allows users to
create, remove, save and update user profiles based on their
recursively changing information requirements and user preferences thus providing a
personalized experience. Preliminary experimental results
indicate that the system produces more user-interest based
results based on explicit user profiles when compared to
traditional search methods which provided the same data for a
similar search to every user.\\

4.\\ \\ A similar study was done and it’s proposed framework
composed of two modules: an off-line module which
pre-processes data to build user and content profiles and predict a recommendation list to
provide more relevant data to users. They applied data mining
techniques to build user profiles, where the prediction of the
user model is accomplished not using explicit user interaction,
but rather implicit information collected from all past usage
sessions which is also known as usage behavior. The input
data for this first step consists mainly of Web server access log
files. In order to extract useful information from log files and
build user profiles, they applied Web Usage-Mining
techniques.They studied the application of data mining techniques to
discover usage sample from data collected, in order to understand
and better serve the needs of its user.\\

5.\\ \\ MICHAEL J. COLE et.al tells the quest for knowledge rooted in complex consumer behaviors resulting
from highly conditionalized decisions. They have mainly focused on development of a user-centered
representation of information search interaction based on the activity patterns of the user during search.
Presentation of a technique to investigate interactions between aspects of user knowledge states that are
not (easily) observed and page discovery and using deliberate user behaviors typically used to describe
search behavior. Results show that task types and task intensity can be differentiated by user activity
pattern properties, in particular by pattern complexity and activity status distribution. They have
performed various kind of analysis activities for personalization such as eye moment analysis , page
sequence analysis ,low level information acquisition activity , clustered page types, complexity of
sequences and Markov State Models and Graph Properties
\vspace{10mm}
\hrule
